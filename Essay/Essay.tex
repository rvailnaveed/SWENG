\documentclass[12pt]{article}
\usepackage[document]{ragged2e}
\setlength{\parskip}{1em}

\title{\textbf{Kevin  Systrom -  Biography}}
\author{Muhammad Rvail Naveed}
\date{}
 
\begin{document}
    \maketitle

    \begin{center}
        \textbf{Module}: Software Engineering - CSU33012 \\
        \textbf{Student Number}: 17321983 
    \end{center}

    \newpage

    Kevin Systrom was born on the 30th December 1983 in Hooliston, Massachusetts. He is 
    an American entrepeneur and Software Engineer. He is best known for creating the monumentally
    popular social media and photo sharing app: Instagram along with his co-founder Mike Krieger. With an estimated net
    worth if \$1.4 Billion, he is included on the list of America's wealthiest individuals under 40.

    Systrom studied Management Science and Engineering in Stanford University 
    where he would later graudate with a Bacelors degree. He spent his third year abroad in Florence, Italy.
    During his time abroad, he studied photography, which would prove to be useful in the future as his deep understanding of photos
    was a major driving force for the intense popularity and user adoption of Instagram.

    While in university, 
    he also got his very first taste of entrepeneurship 
    and startups when he was nominated to participate in the Mayfield Fellows Program. 
    The fellowship led to an internship at a company called Odeo, a podcasting app. The creator of Odeo
    would later go on to create Twitter, interestingly enough Jack Dorsey, the current CEO of Twitter, 
    was also working at Odeo at the same time as Systrom.

    Systrom's desire to become an entrepreneur
    and start his own business started when he was working at Google early in his career. He worked on many google services such as Gmail and Google's own
    Document Suite(Docs, Spreadsheets) etc. He was employed at Google for a total of two years before finally quitting
    due to frustrations of not being promoted to an Associate Product Manager position.

    \section{Early start}
    After his exit from Google he joined a social travel recommendation startup called 
    NextStop. NextStop mixed elements of social recommendations with
    with a reputation system and gameplay. It was bought by Facebook a year after its inception.

    \section{How did that progress}
    During his time at NextStop, Kevin had began work on his own app which he 
    aptly called \textit{Burbn} after  his favourite spirit. The app combined location-based social networking with photo-sharing.
    The app first gained traction when Systrom
    pitched the idea at a party in Silicon Valley, where he found his early adopters as well as two investors who each
    invested \$250,000 even though it was in it's infancy. This was also around the time when Systrom brought on Mike Kreiger as co-founder.

    Although early adopters loved the app, it was a nightmare for newcomers. It was just too clunky. Although initially, Burbn launched
    as a direct competitor to Foursquare, it could not match the apps ease of use. Systrom and Kreiger decided to look at the data,
    and what they found was that people really enjoyed taking pictures, applying a filter from a filter app and posting it for their friends to see. So they completely stripped Burbn of 
    most of it's features, shifting their focus solely to photo-sharing. What was once a competitor to Foursquare had now transformed into a whole new category, with no competition.

    The real "eureka" moment came to Kevin when he was at the beach with his girfriend who was complaing that she didn't want to post
    her pictures online becuase they did not look as his friend Greg's. When Systrom told her that Greg had used a filter app before posting
    his pictures, she told him that he should include filters in Burbn. Thus photo-filters were introduced into the app, the first filter
    created by Kevin himself called "X-Pro II" is still available on Instagram today. Burbn was officially renamed to \textit{Instagram} in
    October 2010.

    \section{Instagram}
    \section{What happened to Instagram}
    \section{Features}
    \section{Views on 'stealing'}
    \section{Leaving Instagram}
    \section{Life now}
    \section{Future Plans}

    \section{Conclusion}
\end{document}