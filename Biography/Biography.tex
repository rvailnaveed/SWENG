\documentclass[12pt]{article}
\usepackage[document]{ragged2e}
\usepackage{biblatex}
\addbibresource{references.bib}
\setlength{\parskip}{1em}

\title{\textbf{Kevin  Systrom -  Biography}}
\author{Rvail Naveed}
\date{}
 
\begin{document}
    \maketitle

    \begin{center}
        \textbf{Module}: Software Engineering - CSU33012 \\
        \textbf{Student Number}: 17321983 
    \end{center}

    \newpage

    Kevin Systrom was born on the 30th December 1983 in Hooliston, Massachusetts. He is 
    an American entrepeneur and Software Engineer. He is best known for creating the monumentally
    popular social media and photo sharing app: Instagram along with his co-founder Mike Krieger. With an estimated net
    worth if \$1.4 Billion, he is included on the list of America's wealthiest individuals under 40.

    Systrom studied Management Science and Engineering in Stanford University 
    where he would later graudate with a Bacelors degree. He spent his third year abroad in Florence, Italy.
    During his time there, he studied photography, which would prove to be useful in the future as his deep understanding of the subject
    was a major driving force for the intense success of Instagram.

    While in university, 
    he also got his very first taste of entrepeneurship 
    and startups when he was nominated to participate in the Mayfield Fellows Program. 
    The fellowship led to an internship at a company called Odeo, which specialised in allowing users to
    create, record, and share podcasts with a simple Adobe Flash-based interface.
    The creator of Odeo
    would later go on to create Twitter. Interestingly enough, Jack Dorsey, the current CEO of Twitter, 
    was also working at Odeo at the same time as Systrom.

    Systrom's desire to become an entrepreneur
    and start his own business started when he was working at Google early in his career. He worked on Google services such as Gmail and Google's own
    Document Suite(Docs, Spreadsheets). He was employed at Google for a total of two years before finally quitting
    due to frustrations of not being promoted to an Associate Product Manager position.

    After his exit from Google he joined a startup that specialised in social travel recommendations called 
    NextStop. NextStop mixed elements of social recommendations
    with a reputation system and gameplay. It was bought by Facebook, who shut it down, a year after its inception.

    While at NextStop, Kevin had began work on his own app which he 
    aptly called \textit{Burbn} after  his favourite spirit. The app combined location-based social networking with photo-sharing.
    The app first gained traction when Systrom
    pitched the idea at a party in Silicon Valley, where he found early adopters as well as two investors who each
    invested \$250,000 even though the app was in it's infancy. This was also around the time when Systrom brought on Mike Kreiger as co-founder.

    Although early adopters loved the app, it was a nightmare for newcomers. The clunky and overbearing UI proved too steep a learning curve for many. 
    Although initially, Burbn launched
    as a direct competitor to Foursquare, it could not match the apps ease of use. Systrom and Kreiger decided to look at the data,
    and what they found was that people really enjoyed taking pictures, applying a filter from a filter app and posting it for their friends to see. So they completely stripped Burbn of 
    most of it's features, shifting their focus solely to photo-sharing. What was once a competitor to Foursquare had now transformed into a whole new category, with no competition.

    The real "eureka" moment came to Kevin when he was at the beach with his girfriend who was complaing that she didn't want to post
    her pictures online becuase they did not look as his friend Greg's. When Systrom told her that Greg had used a filter app before posting
    his pictures, she told him that he should include filters in Burbn. Thus photo-filters were introduced into the app, the first filter
    created by Kevin himself called "X-Pro II" is still available on Instagram today. Burbn was officially rebranded to \textit{Instagram}, (a combination of "instant" and "telegram") in
    October 2010.

    When Instagram first launched, more than 25,000 people downloaded it in the first 24 hours causing it's servers to crash.
    During its first couple of years, the app saw exponential growth amassing an excess of 50 million users in 2012. This was even more impressive
    considering that Instagram only employed 13 people including Systrom and Kreiger at the time of it's acquisition by Facebook.

    Systrom's excellent understanding of user behaviour from Burbn coupled with his passion and knowledge of photographty greatly boosted the success of Instagram.
    The distinct square photos, simplicity, ease of use and built in filters meant that users could easily adopt the app and not jump between other apps to edit their photos
    before posting kept users locked in and coming back. The uniqueness of the app compared to competitors felt fresh compared to longstanding rivals such as Facebook and that also drew people in.

    In 2012, Instagram was bought by social media giant, Facebook, for \$1 Billion. This was a notable move for Facebook, who up until now had only focused on smaller acquisitions,
    worth less than \$100 million.
    Systrom, who owned 40\% of the company at the time,
    reportedly made a total of \$400 million from this deal. This acquisition was extremely beneficial to Facebook who were 
    struggling in the mobile market but now had a foothold through Instagram. 
    Mark Zuckerberg said at the time that he believed Instagram and Facebook were seperate, unique entities 
    and that Instagram would continue to operate independantly of its now parent company.

    Under Facebook, the company thrived and implemented many new groundbreaking features. One of the more controversial ones being \textit{Instagram Stories}, which Systrom admits is a copy of the
    popular \textit{Snapchat Stories}. Many people think that Instagram Stories was a direct response by Facebook to Snapchat when they denied the offer of acquisition by Facebook.
    Systrom has publicly defended the feature, insisitng that all new services launched by tech companies are "remixes" of other products and that
    "all of these ideas are original when you remix them and bring your own flavour". This move crippled Snapchats user base as creators, influencers and average users alike realised how limited
    Snapchat's version was and how much more people could be reached using Instagram's alternative. 

    Despite the success of Instagram, there were rumours that 
    Kevin and his co-founder Mike Kreiger were experiencing 
    creative differences with their parent company, Facebook's CEO; Mark Zuckerberg. Following these rumours, Systrom abruptly
    announced that he would be leaving Instagram alongside Kreiger on September 24th 2018. Systrom's resignation from the company was viewed 
    by many as the end of Instagram's pseudo-independance from it's parent company, Facebook.
    Although Systrom has refused to comment on his departure from the company he has said that "no one ever leaves a job because everything's awesome"
    and that he had wrote his resignation to Zuckerberg "very, very quickly", indicating that there were indeed tensions between the two.
    
    After leaving Instagram, Systrom vowed to create another company, although he has not done that as of yet. He 
    has lived a relatively quiet life since his exit from Instagram. In 2015 He married his girlfriend, 
    a fellow graduate of Stanford University, whom he now shares a daughter with.
    
    Recently he has invested  \$30 million in a new startup \textit{Pitch} which was created by the people who developed Wunderlist.
    Wunderlist was sold to Microsoft in 2015 for \$100 - \$200 million. Pitch aims to be a direct competitor to Microsoft's PowerPoint software.
    This could mark the start of Systrom's return to the software industry, but one thing is for certain, Silicon Valley has not yet seen the end of Kevin Systrom.

    \medskip
 
    \begin{thebibliography}{9}

    \bibitem{rise} 
        The career rise of Instagram CEO Kevin Systrom - Business Insider   
        \\\texttt{http://tiny.cc/jru3dz}
        \\
        \bibitem{wiki}
        Kevin Systrom - Wikipedia        
        \\\texttt{http://tiny.cc/d0u3dz}
        \\
    \bibitem{nextstop} 
        Facebook acquires NextStop - techcrucnh
        \\\texttt{http://tiny.cc/o8u3dz}
        \\
    \bibitem{growth}
        Growth Lessons from Instagram        
        \\\texttt{http://tiny.cc/hdv3dz}
        \\
    \bibitem{buys}
        Facebook buys Instagram - dealbook      
        \\\texttt{http://tiny.cc/dgv3dz}
        \\
    \bibitem{quits}
        Kevin Systrom quits Instagram - The Verge
        \\\texttt{http://tiny.cc/tkv3dz}
        \\
    \bibitem{timeline}
        Timeline of Instagram - Wikipedia    
        \\\texttt{http://tiny.cc/rmv3dz}
        \\
    \bibitem{timeline}
        Systrom invests in Pitch - techcrunch
        \\\texttt{http://tiny.cc/fpv3dz}
        \\
    \end{thebibliography}

\end{document}